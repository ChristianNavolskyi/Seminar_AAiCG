\chapter{Summary}
\label{ch:summary}
Karl\'ik et al.~\cite{Karlik2019} extended the options for variance reduction in the MIS domain
by introducing a technique to modify one of the sampling densities to better fit the integrand.
They derived a MIS-compensated pdf from the standard balance heuristic in an MIS setup to reduce variance.
Practical more simple pdf were presented for use in current renderers
and also a optimal pdf which is difficult to create.
Examples in image based lighting and path guiding were shown,
were the advantage in IBL is much more substantial than in path guiding they could still improve both applications of MIS.


\section{Future Work}
\label{sec:future_work}
They used a diffuse BRDF for their approximations and therefore the results excel when working with diffuse surfaces,
but don't help much when the BRDF becomes more specular.
Therefore approximations that incorporate the BRDF could further reduce variance.

In IBL the used unoccluded illumination which already helped a lot,
but taking occlusion into account could also further improve results.

In their tests they used equal samples among all pdfs.
Combining their approach with also optimally distributing samples over the used pdfs could additionally help with variance reduction.