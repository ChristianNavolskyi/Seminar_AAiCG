\chapter{Related Work}
\label{ch:related_work}

The MIS framework was introduced in~\cite{veach_guibas}
and combines multiple different sampling techniques with a weighting function to sample a more complex function better.
In their work they proposed a few different weighting functions the most famous being the balance heuristic,
but also the power and cutoff heuristic which perform better in certain scenarios where one techniques matches the integrand well.
Despite the balance heuristic being provably good Kondapaneni et al.~\cite{Kondapaneni2019} developed a weighting function that is truly optimal
and uses negative weights to archive even better results.\\
A lot of work has been put into optimizing the weighting functions,
but Karl\'ik et al. followed a different approach
and worked on a technique to adjust one sampling density to reduce the variance.
\todo{Last sentence in first paragraph}

Another direction of research in the field of MIS is to figure out how to distribute the samples over the different sampling techniques.
Pajot et al.~\cite{pajot} proposed a method where they calculate a so called representativity based on common rendering information such as the BRDF,
which is a measure of how well a technique matches the integrand.
In a one-sample estimator this representativity can also be used to derive a probability for each sampling strategy.\\
Lu et al.~\cite{lu} created a method where no prior knowledge about the scene is needed to generate a sample distribution over two sampling strategies.
They start by sampling with a small number of samples evenly distributed over the strategies
and calculate a value to partition the total number of samples over the sampling techniques.\\
He and Owen~\cite{he} presented another method to get a partitioning of samples for more than two strategies
by proving that the variance is jointly convex in the distribution of samples.
They also proposed an improved method for convex optimization to find such a distribution.\\
Sbert et al.~\cite{sbert} worked on a modification of the balance heuristic that doesn't take the distribution of samples into account for calculating the weights.
Their method works in two phases.
The first phase uses 20\% of the total samples equally distributed over all sampling methods to calculate the variance $ \sigma^2_i $.
After that the following stage is subdivided into eight substages each using 10\% of the total number of samples
and iteratively updating $ \sigma^2_i $ and calculating the distribution $ \alpha $ over the techniques for the next substage.\\
What all the above techniques have in common is that they all rely on initial samples to fine tune their succeeding execution.
The proposed method of Karl\'ik et al. on the other had doesn't need any prior samples to be taken and therefore their performance impact should be much smaller~\cite{Karlik2019}.

One widely used application of MIS is image based lighting with BSDF and HDR map sampling~\cite{pbr-book}.
When using a HDR environment map for direct illumination the pdf for that map is usually proportional to the brightness of the HDR map,
Karl\'ik et al.~\cite{Karlik2019} created a method that alters that sampling density to further decrease the variance.
Agarwal et al. came up with the idea of stratifying the environment map
and taking into account the visibility in the scene
to create a method that can require up to two orders of magnitude less samples for the same quality~\cite{agarwal}.\\
Another approach from Clarberg and Akenine-Möller uses product importance sampling
and creating the BSDF on the fly during rendering.
Their precomputation step uses quadtree-based multiplication while during rendering they build the approximation of the BSDF
and evaluate the product in only one single tree traversal which makes it perform faster than other methods~\cite{clarberg}.\\
The method of Karl\'ik et al. does not change the sampling procedure itself and therefore can work without modifications of the environment map sampler~\cite{Karlik2019}.




\todo{Sum up what others did in this field.}