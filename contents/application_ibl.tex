\chapter{Application: Image Based Lighting}
\label{ch:application_ibl}
In this chapter we start by describing image based lighting shortly.
Following that the approach of Karl\'ik et al. is explained to use in image based lighting.
\todo{Explain the contents of this chapter (at the end)}


\section{Image Based Lighting}
\label{sec:ibl}
Creating realistic lighting for a scene can be very complex and time consuming.
To avoid modelling a lighting environment a commonly used method is to create an image of a real world environment
and using that for the direct illumination of a scene.
This approach is called image based lighting and was introduced back in 1998 by Debevec in~\cite{debevec}.\\
To get the lighting of a scene in the real world you could place a mirror sphere in the middle of your scene
and the a photo of that sphere ideally from far away
so the camera itself doesn't take up a large part of the scene.
The name of this method is sphere mapping.
During rendering you place your scene in the center of that textured sphere
and if a shadow ray doesn't hit any object of the scene you read out the texel
on the sphere that corresponds to the direction of the ray.\\
A different method is to take six pictures of the scene from the top, bottom, left, right, front and back
and then use the same evaluation method as with the sphere with the direction of the shadow ray direction
to get the texel from the environment map.
For more information on how to use environment maps/image based lighting consider~\cite{environment_map}.
\todo{Information about sampling the environment map}


\section{MIS Compensation in Image Based Lighting}
\label{sec:misc_ibl}
\todo{Explain how their approach improves ibl variance}


\todo{If space left show results}