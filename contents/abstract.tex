\chapter{Abstract}
\label{ch:abstract}
In this work we give an overview about multiple importance sampling (MIS)
and in more detail focus on MIS compensation a technique introduced by Karl\'ik et al. in~\cite{Karlik2019}.
Their idea is to pick one of the multiple sampling techniques available in MIS framework
and modify it in a way that it compensates for the averaging effect introduced through the balance heuristic.
We present their example applications in image based lighting
and path guiding~\cite{mueller2017} and go over their results to show the effectiveness of MIS compensation.

\todo{Is that enough for the abstract? In general the abstract is relatively short.}


\todo{Write abstract (at the end)}

\todo{Write introduction. Structure as in the paper (overview, problem, others, solution proposal)}

\todo{Citation before the dot. In more than one scentance, name in the first sentence and referenc in the last sentence.}

\todo{are all the citations correct?}

\todo{Does the date of access need to be in the references?}

\todo{Fig. 6 in the paper, how can multiple samples be drawn from the one-sample estimator?}