\chapter{Application: Path Guiding}
\label{ch:application_pg}
In this chapter we will give a short overview of what path guiding is.
After that we will look at how Kerl\'ik et al.~\cite{Karlik2019} applied MIS compensation on path guiding to further improve render quality.


\section{Path Guiding}
\label{sec:path_guiding}
The ides behind importance sampling is to let the sampling density be proportional to the integrand
to get a small as possible variance in the result or image in our use case.
Normally we can use the BRDF and the direct illumination to create two sampling distributions,
but one important part is the indirect illumination which is not known beforehand and therefore can't be used easily.
That's exactly what path guiding is for.
It learns the incident radiance and builds a distribution based on this.
This distribution is then used to create a new image from scratch to create a better distribution
until the final iterations yields the image after combining all iterations weighted by their inverse variance~\cite{Vorba_2019}.
During rendering the distribution can be normally used in MIS.


\section{MIS Compensation in Path Guiding}
\label{sec:misc_path_guiding}
\todo{Explain how their approach improves path guiding}