\chapter{Conclusion}
\label{ch:conclusion}
Karl\'ik et al.~\cite{Karlik2019} extended the options for variance reduction in the MIS domain
by introducing a technique to modify one of the sampling densities to better fit the integrand.
They derived a MIS compensated pdf from the standard balance heuristic in an MIS setup to reduce variance.
A practical and simpler pdf was presented for use in current renderers
and also an optimal pdf that is difficult to create but shows what is possible.
Examples in image-based lighting and path guiding were shown,
where the advantage in IBL is much more substantial than in path guiding,
but they could still improve both applications of MIS.


\paragraph{Future Work}
\label{par:future_work}
\textit{Specular BRDF} Karl\'ik et al.~\cite{Karlik2019} used a diffuse BRDF for their approximations
and therefore the results excel when working with diffuse surfaces,
but don't aid much when the BRDF becomes more specular.
Therefore approximations that incorporate the BRDF could further reduce variance.

\textit{Image-based lighting} In IBL they used unoccluded illumination which already improved the results a lot,
but taking occlusion into account could further reduce variance.

\textit{Combining compensation and sample distribution} In their tests they used equal samples among all pdfs.
Combining their approach with also optimally distributing samples over the used pdfs could additionally help with variance reduction.